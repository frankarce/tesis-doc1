\chapter{Introduction} \label{introduction} 

The purpose of this research is to contribute in the diverse forms of use of the interactive learning environmentsby proposing a learning environment, we based the content management with an adaptive hypermedia approach and 
with the development of a new type of learning object to be adapted to the learning environment.
This environment use various devices that were capable of running a web browser, non-relational data bases for information exchange and some sensors like cameras and Kinect 2 were used. In addition we implemented a way to predict the level of user attention
which it was compared against information obtained by a video taken from the user doing the activity.

\section{Motivation}
There are many types of learning environments starting from the most basic that exist almost from the beginning of civilization where a person is able to learn based on their current context.
For example the first humans who needed to hunt them could take decisions and adapt to the situation and perform the task that was to hunt an animal to obtain food, in more recent times we can relate to the classrooms of schools, and currently including technology these Environments can be configurable and adapt to a particular context (the user).When a user uses learning environments the themes are usually flat and have the same content for everyone. In addition, users think and assimilate information in a different way which makes it more attractive to have a learning environment that adapts to the learning styles of users and why not to their preferences.

Nowadays most interactive museums work stations or exhibits where users come to the station to interact with or receive information by reversing some time on it until it passes to the next station, where the display will probably showed relevant information for the person but if this information is not shown to digestible way (processed so that it is attractive to the user) the user will probably spend less time or not time at all at the station. This expose the lack of adaptation of the exhibits in some interactive museums or standard museums. in order to ensure that the information and how is presented to the user is broadly engaging. 
Intelligent learning environments can be used as exhibitors in museums they use embedded systems, sensors, information and communication technologies that are becoming invisible to the user as they are being integrated into physical objects, infrastructure, the environment in which we live, work and many other environments. This idea provides a good way of bridging the gap between human users and computing systems, and this motivates related research into Computing. Some of these systems use learning resources called learning objects. For the ex-change of learning objects between systems standardization initiatives have been developed and there are some implementations and repositories that manage the content using these standards. 

\section{Learning Enviroment}

Learning environments are important in the day-to-day lives of people because we are in contact with them all the time in accordance with Phillips\cite{PhilMcNaKenn2010zx} a Learning Environment(LE) is a place where resources, time and reasons are available for a group of people to nurture, support and value the learning of a limited set of information. The LE are social places even when only one person is found there. One of the challenges facing the design of learning environments is human complexity, because each person thinks and assimilate information in different ways making it difficult to identify which resources are adequate for everyone. Intelligent learning environments (ILE) are a new type of intelligent educational system, which combines characteristics of traditional intelligent tutoring systems (ITS) \cite{john1991} and learning environments. According to Self \cite{self1998} ITS are learning systems based on computers that try to adapt to the needs of the learner. 


\begin{figure}[h!]  
\centering  
\includegraphics[]{pizzarra}
\quad  
\caption{title}  
\label{name}  
\end{figure}


%\includegraphics[natwidth=162bp,natheight=227bp]{pizzarra.png}
%\includegraphics[natwidth=162bp,natheight=227bp]{img/pizzarra.png}


  
 

\section{Learnign Objects.}

\section{Aim's.}
\section{Outline.}