\chapter{Art State} \label{artstate} 
In this section the state of the art in intelligent learning environments, e-learning and context are presented. The learning environments have been 
(Falta aqui)

\section{E-learnign}
Aprendizaje colaborativo basado en recursos adaptativos.

This work consists of an Adaptive Hypermedia System \cite{Valdez2007}, this work is focused on semi-automatic sequencing of educational resources. It builds on information that can be subjective; for example, the user's knowledge, learning style, their predilections and even evaluation; they are perceived differently depending on the context. Sequenced personalized teaching materials; using fuzzy rules and attributes to treat subjective information, involved in the learning process. I n a hypermedia system, such as the Web (W3C, 2008), users surf by  a network of resources for information or to perform some task. In the case of a student, navigation can be part of a complex learning activity; an activity may consist of: search some information, discuss the findings with other students (online), and then make a presentation where resources of different formats and sources are integrated.
In this paper propose: the architecture for a learning environment based on reusable resources, using techniques such as Adaptive Hypermedia and engine adaptation an extension to Simple Sequencing specification where a particular system of fuzzy inference. In include the use of systems fuzzy inference can facilitate the damnation of rules by the instructors, because in the learning process involved usually fuzzy variables.

\section{Learning Environments}
\section{Context}
PCULS
“Personalized context-aware ubiquitous learning system for supporting effective English vocabulary learning” [Chen 2010]. This work consists of a system that supports students in English using a situational approach to learning based on the location of the learner. Detected by wireless positioning techniques, learning time, individual skills of students in English and free time, which enables students to fit your learning content to effectively support the learning of English in a school environment generating vocabulary appropriate to the situation and presenting textual information via mobile phones. When only text information shown this lack of context that was trying to capture in the vocabulary.
 

Procedure system operations:
In Step 1: The learner enters the system through the user interface where the system checks the user's account and free time available to the user.
In Step 2: After the apprentice enters the system, the student agent location automatically determines its location using a location technique based on neural networks.
In step 3 and 4: Based on the location of the learner, the context analyzer agent receives contextual information database user and context. Therefore appropriate agent finder English vocabulary seeks the context of the learner according to the analytical results of context analysis agent fits.
In step 5: Content delivery agent organizes the English learning materials discovered by the search agent material learning English as appropriate content and transmits it to the device of the learner.
In step 6: The message delivery agent transmits the learning content to the learner PDA as a web page or in the form of short message to the cell apprentice. then the trainee returns to step 2 to start the next learning cycle or exit the system, completed the learning process.
The application was tested with two sets of users. they applied a test for each user group with these results.
 
	
Display-based services through identification: An approach in a conference context.
This work combines ubiquitous computing, context, information display and [Hervas 2006] devices, to provide services to the user implicitly, with little or no user intervention feeding context information provided by sensors embedded in devices the environment. Its primary objective is to obtain service information display adaptable to changes in context, providing the same information. They base their research on the identification of users, knowing their profile, their situation in a given underlying information and much other time. Which results in a system that identifies users through RFID cards, and after analyzing generated user information service based on the information display. The whole operation of the service information display is divided into three distinct modules: Context Analysis, generation module mosaics and Composer module. The contextual model represented by an ontology proposed by the authors. The following figure shows the outline of the project is shown.
 
Every moment is available context information filtered according to the ontology and obtained from the system database and embedded sensors in the context (in principle RFID antennas and RFID tags information). The context analysis module is responsible for deciding when a change in the context must lead to changes in display devices. It is also responsible for the system database maintains the consistency of information on these changes. mosaics are generated after an analysis of the background information so that the appropriate and optimal environmental conditions to display information. The generator module provides an XML description of the Mosaic module compounder. From the description of the specific information to be displayed and the mosaic is generated is obtained, publishing through the devices of the environment and taking into account their differences.

 
