\chapter{Conclusions and future work} \label{conclu} 
%En este trabajo de investigaci�n, se desarrollo un ambiente de aprendizaje interactivo con m�ltiples usos como exhibidor de museo, entretenimiento, publicidad, etc. La principal caracter�stica de este ambiente de aprendizaje es que cuenta con un objeto de aprendizaje especial que al combinarse con el ambiente le permite ser lo que se mencionaba antes.
%El desempe�o de este ambiente lo medimos en varias pruebas con usuarios reales, el ambiente en sus primeras versiones fue probado con usuarios �locales� (Usuarios al alcance del laboratorio) para medir la aceptaci�n de este tipo de tecnolog�a y que tan f�cil era de utilizar el ambiente en la cual tuvimos buenos resultados.
In this research work, an interactive learning environment was developed with multiple uses as a museum exhibitor, entertainment, advertising, etc. The main feature of this learning environment is that it has a special learning object that when combined with the environment allows it to be what was mentioned before.
The performance of this environment is measured in several tests with real users, the environment in its early versions was tested with "local users" (Users within the scope of the laboratory) to measure the acceptance of this type of technology and how easy it was to use The environment in which we had good results.
 
%En la segunda versi�n nos trasladamos al museo interactivo de Tijuana El trompo donde el objetivo de la prueba fue usarlo con usuarios de diversas edades y observar las reacciones de estos mismos desde esta versi�n utilizamos el sensor de Kinect 2 para observar el usuario pero la informaci�n obtenida no fue lo suficientemente buena para utilizarla, en esta ocasi�n el ambiente con informaci�n dada por el usuario era capaz de modificar la informaci�n mostrada en un intento por adaptar la informaci�n y ser m�s atractiva para el usuario. Al no poder utilizar la informaci�n dada por el sensor optamos por conservar las encuestas que les aplicamos a los usuarios las cuales mostraron muy buena aceptaci�n en su mayor�a, los pocos que no estuvieron de acuerdo fueron ni�os menores de 6 a�os los cuales mostraron poco inter�s y un nivel de distracci�n un poco alto esto lo atribuimos a que los ni�os eran extranjeros y no dominaban muy poco idioma espa�ol que era el idioma que manejaba el ambiente. 
In the second version we moved to the interactive museum in Tijuana The spinning wheel where the objective of the test was to use it with users of different ages and to observe the reactions of these same ones from this version we used the sensor of Kinect 2 to observe the user but the obtained information Was not good enough to use it, this time the environment with information given by the user was able to modify the information shown in an attempt to adapt the information and be more attractive to the user. When we could not use the information given by the sensor we chose to keep the surveys we applied to the users, which showed very good acceptance in the majority, the few who did not agree were children under 6 years old who showed little interest and A level of distraction a little high we attribute this to the children were foreigners and did not have very little Spanish language that was the language that managed the environment.  

%En la �ltima versi�n del ambiente fue utilizado en el instituto tecnol�gico de Tijuana utilizando un grupo de usuarios m�s homog�neo los cuales eran alumnos y algunos maestros de la escuela, en esta prueba los usuarios adem�s de ser observados por el sensor tambi�n se les tomo video para analizarlo y tener una referencia para los resultados de la clasificaci�n de datos dada por el sensor adem�s se le aplico una encuesta a los usuarios para identificar los gustos en cuanto a informaci�n y los dispositivos que utilizan los usuarios. cada una de las pruebas el ambiente evolucionaba gracias a la informaci�n proporcionada por los usuarios de ir de algo muy simple como solo mostrar im�genes a mostrar videos, reproducir audios. Luego de ciertas modificaciones pudimos manipular contenido y secuenciarlo.
In the last version of the environment was used in the technological institute of Tijuana using a more homogeneous group of users who were students and some teachers of the school, in this test users in addition to being observed by the sensor were also taken video for Analyze it and have a reference for the results of the classification of data given by the sensor also applied a survey to users to identify the tastes in terms of information and devices used by users. Each of the tests the environment evolved thanks to the information provided by users to go from something very simple like just showing pictures to show videos, play audio. After some modifications we were able to manipulate content and sequence it.
