\chapter{The proposed framework} \label{proposmeth} 
In this chapter, an architecture for developing interactive learning environments is proposed. The architecture allows the use of techniques of adaptive hypermedia systems and use in interactive environments. This architecture has as resources environmental learning objects, which are presented to users through the various devices that can be part of this environment and are organized by a sequence based on Simple Sequencing. In Figure X you can see the components of this architecture. As this environment has multiple configurations of use, sometimes it can help a fuzzy inference system to adapt the information presented by the environment to users. The components of the architecture are described in detail.
\section{ELO}
A standard learning object is defined in [2]. The teacher prepares teaching materials using content from various sources; select different pieces of information that subsequently assembled to form the course or class to teach. Learning objects are based on this methodology, seeing the didactic content as a component that is designed to combine with others and to be used in different contexts; learning objects usually have the following characteristics: 
�	They are self-contained. Each learning object can be used independently. 
�	Are reusable. A learning object can be used in different contexts, for multiple purposes. 
�	Can be added. Learning objects can be grouped into collections, after being presented with a traditional course structure. 
�	Are labeled with metadata. Each learning object has associated certain in-formation that describes it. This facilitates reuse by automatic means.
We propose a learning environment that uses a new type of learning object which we call "environmental learning object" (ELO) these objects include additional metadata that can be used to identify and manage content for their use in ubiquitous devices: we considered input and output devices e.g. interactive tables, wall projections, monitors, tablets, cameras, microphones and speakers.
By tagging through the metadata we can make a selection of learning objects to create and manage different contexts when applied to a learning environment as we can note in Figure.
Where each learning object is selected and routed to a device that operates its utility like a sound file that can�t be used on a monitor maybe on a Smartphone or a tablet pc, these are the types of problems that can occur when using these learning objects. Also this ELO will give contextual information for feedback. Then the learning object is attached to the sequence as we can see on the figure 3.1.
\section{Fuzzy simple sequencing}
Simple sequencing standard[18] [19], which allows sequencing of this type of learning objects. First an activity tree with n number of activities is determined by the teacher, selecting the learning objects to be used per activity and adding fuzzy pre-condition rules that consider the context and user model to determine the sequence, for example:
IF Context.Temperature is HIGH and Session.Activity.Place is OUTSIDE
THEN
this.Precondition. = SKIP.
Then context information is obtained from users using the sensors mentioned above, each activity will have 1 or more environmental learning objects, thanks to the metadata included in the learning objects will be displayed in the appropriate device for example if we have a web page can be displayed on a monitor, a questionnaire on a tablet, a video projected on a wall, sound on speakers, an interactive game on the table as we can see in Figure 3.
The main components of the Simple Sequencing standard are the Learning Activities and the Activity Tree. A Learning Activity is defined as a pedagogically neutral unit of instruction, knowledge, evaluation, etc. Learning Activities can have nested sub-activities arbitrary depth. 
There is an implicit hierarchy of containers in the tree. Depending on the application concept labels can be applied to learning activities. Only leaf nodes can be associated with Resources Activity (the equivalent of Learning Objects).
An example of the configuration obtained by sequencing can be observed in Figure 3.2. The tree is traversed as follows from the root there is the General activity (can be any subject in particular) with 2 nodes in it, we see that has the attribute "forwardOnly" on true, this means they have to be traveled sequentially by users, the first activity is a pre-evaluation, which is associated with a learning object in this case a test that contains a rule that makes the activity is carried out or otherwise will not advance to the next activity, as specified in the Pre_condition rule.
When a learning sequence is generated learning activities are established and they not change, what changes is the action as if you skip, performed, jump, hide, repeat or show an activity, this decision is taken by the rules of precondition. In this paper we seek to generate diversity of resources displayed in the activities adapting the resources of the activity to the user,  for example we have an introductory activity of a particular subject and 2 users of different ages perform the activity, the first user of 25 years will be shown more detailed and textual information, while the second user of 6 years of age is in a very early stage of learning resources that show you are easier to understand and have more audio-visual con-tent. With fuzzy logic generate inputs for use in the precondition rules for exam-ple: fuzzy rule if user.age is low and user.academic_level is low then user.level is begginer. On the precondition rule we will use the output of the fuzzy rule will use the output of that fuzzy rule as follows if user=begginer then pre_condition = show. Thus a set of resources is obtained for a novice user and these are shown on the right devices for each resource through a web browser running in this device as we can see on the figure 1.

Los recursos son los objetos de aprendizaje, los cuales como ya sabemos pueden ser cualquier video, imagen, audio o texto, para esta investigaci�n utilizamos una gran cantidad de im�genes videos y audios. Con un aproximado de 100 im�genes, 10 videos y m�s de 20 audios fue lo que formo parte de cada una de las configuraciones del ambiente. Al inicio las dimensiones de las im�genes fueron un problema ya que no todos los dispositivos ten�an resoluciones similares, tiempo despu�s se implemento una forma de ajustar las im�genes a la resoluci�n de cada pantalla o proyector. Al igual que las im�genes los videos tuvieron el mismo problema ya solucionado en las �ltimas versiones.

\section{Devices}
\section{Master / User}